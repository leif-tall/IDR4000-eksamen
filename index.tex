% Options for packages loaded elsewhere
\PassOptionsToPackage{unicode}{hyperref}
\PassOptionsToPackage{hyphens}{url}
\PassOptionsToPackage{dvipsnames,svgnames,x11names}{xcolor}
%
\documentclass[
  letterpaper,
  DIV=11,
  numbers=noendperiod]{scrreprt}

\usepackage{amsmath,amssymb}
\usepackage{iftex}
\ifPDFTeX
  \usepackage[T1]{fontenc}
  \usepackage[utf8]{inputenc}
  \usepackage{textcomp} % provide euro and other symbols
\else % if luatex or xetex
  \usepackage{unicode-math}
  \defaultfontfeatures{Scale=MatchLowercase}
  \defaultfontfeatures[\rmfamily]{Ligatures=TeX,Scale=1}
\fi
\usepackage{lmodern}
\ifPDFTeX\else  
    % xetex/luatex font selection
\fi
% Use upquote if available, for straight quotes in verbatim environments
\IfFileExists{upquote.sty}{\usepackage{upquote}}{}
\IfFileExists{microtype.sty}{% use microtype if available
  \usepackage[]{microtype}
  \UseMicrotypeSet[protrusion]{basicmath} % disable protrusion for tt fonts
}{}
\makeatletter
\@ifundefined{KOMAClassName}{% if non-KOMA class
  \IfFileExists{parskip.sty}{%
    \usepackage{parskip}
  }{% else
    \setlength{\parindent}{0pt}
    \setlength{\parskip}{6pt plus 2pt minus 1pt}}
}{% if KOMA class
  \KOMAoptions{parskip=half}}
\makeatother
\usepackage{xcolor}
\setlength{\emergencystretch}{3em} % prevent overfull lines
\setcounter{secnumdepth}{5}
% Make \paragraph and \subparagraph free-standing
\ifx\paragraph\undefined\else
  \let\oldparagraph\paragraph
  \renewcommand{\paragraph}[1]{\oldparagraph{#1}\mbox{}}
\fi
\ifx\subparagraph\undefined\else
  \let\oldsubparagraph\subparagraph
  \renewcommand{\subparagraph}[1]{\oldsubparagraph{#1}\mbox{}}
\fi


\providecommand{\tightlist}{%
  \setlength{\itemsep}{0pt}\setlength{\parskip}{0pt}}\usepackage{longtable,booktabs,array}
\usepackage{calc} % for calculating minipage widths
% Correct order of tables after \paragraph or \subparagraph
\usepackage{etoolbox}
\makeatletter
\patchcmd\longtable{\par}{\if@noskipsec\mbox{}\fi\par}{}{}
\makeatother
% Allow footnotes in longtable head/foot
\IfFileExists{footnotehyper.sty}{\usepackage{footnotehyper}}{\usepackage{footnote}}
\makesavenoteenv{longtable}
\usepackage{graphicx}
\makeatletter
\def\maxwidth{\ifdim\Gin@nat@width>\linewidth\linewidth\else\Gin@nat@width\fi}
\def\maxheight{\ifdim\Gin@nat@height>\textheight\textheight\else\Gin@nat@height\fi}
\makeatother
% Scale images if necessary, so that they will not overflow the page
% margins by default, and it is still possible to overwrite the defaults
% using explicit options in \includegraphics[width, height, ...]{}
\setkeys{Gin}{width=\maxwidth,height=\maxheight,keepaspectratio}
% Set default figure placement to htbp
\makeatletter
\def\fps@figure{htbp}
\makeatother
\newlength{\cslhangindent}
\setlength{\cslhangindent}{1.5em}
\newlength{\csllabelwidth}
\setlength{\csllabelwidth}{3em}
\newlength{\cslentryspacingunit} % times entry-spacing
\setlength{\cslentryspacingunit}{\parskip}
\newenvironment{CSLReferences}[2] % #1 hanging-ident, #2 entry spacing
 {% don't indent paragraphs
  \setlength{\parindent}{0pt}
  % turn on hanging indent if param 1 is 1
  \ifodd #1
  \let\oldpar\par
  \def\par{\hangindent=\cslhangindent\oldpar}
  \fi
  % set entry spacing
  \setlength{\parskip}{#2\cslentryspacingunit}
 }%
 {}
\usepackage{calc}
\newcommand{\CSLBlock}[1]{#1\hfill\break}
\newcommand{\CSLLeftMargin}[1]{\parbox[t]{\csllabelwidth}{#1}}
\newcommand{\CSLRightInline}[1]{\parbox[t]{\linewidth - \csllabelwidth}{#1}\break}
\newcommand{\CSLIndent}[1]{\hspace{\cslhangindent}#1}

\KOMAoption{captions}{tableheading}
\usepackage{fontspec}
\setmainfont{Times New Roman}
\renewcommand{\normalsize}{\fontsize{12}{18}\selectfont}
\makeatletter
\makeatother
\makeatletter
\@ifpackageloaded{bookmark}{}{\usepackage{bookmark}}
\makeatother
\makeatletter
\@ifpackageloaded{caption}{}{\usepackage{caption}}
\AtBeginDocument{%
\ifdefined\contentsname
  \renewcommand*\contentsname{Table of contents}
\else
  \newcommand\contentsname{Table of contents}
\fi
\ifdefined\listfigurename
  \renewcommand*\listfigurename{List of Figures}
\else
  \newcommand\listfigurename{List of Figures}
\fi
\ifdefined\listtablename
  \renewcommand*\listtablename{List of Tables}
\else
  \newcommand\listtablename{List of Tables}
\fi
\ifdefined\figurename
  \renewcommand*\figurename{Figur}
\else
  \newcommand\figurename{Figur}
\fi
\ifdefined\tablename
  \renewcommand*\tablename{Tabell}
\else
  \newcommand\tablename{Tabell}
\fi
}
\@ifpackageloaded{float}{}{\usepackage{float}}
\floatstyle{ruled}
\@ifundefined{c@chapter}{\newfloat{codelisting}{h}{lop}}{\newfloat{codelisting}{h}{lop}[chapter]}
\floatname{codelisting}{Listing}
\newcommand*\listoflistings{\listof{codelisting}{List of Listings}}
\makeatother
\makeatletter
\@ifpackageloaded{caption}{}{\usepackage{caption}}
\@ifpackageloaded{subcaption}{}{\usepackage{subcaption}}
\makeatother
\makeatletter
\@ifpackageloaded{tcolorbox}{}{\usepackage[skins,breakable]{tcolorbox}}
\makeatother
\makeatletter
\@ifundefined{shadecolor}{\definecolor{shadecolor}{rgb}{.97, .97, .97}}
\makeatother
\makeatletter
\makeatother
\makeatletter
\makeatother
\ifLuaTeX
  \usepackage{selnolig}  % disable illegal ligatures
\fi
\IfFileExists{bookmark.sty}{\usepackage{bookmark}}{\usepackage{hyperref}}
\IfFileExists{xurl.sty}{\usepackage{xurl}}{} % add URL line breaks if available
\urlstyle{same} % disable monospaced font for URLs
\hypersetup{
  pdftitle={IDR4000 Mappeeksamen},
  pdfauthor={Kandidatnr.},
  colorlinks=true,
  linkcolor={blue},
  filecolor={Maroon},
  citecolor={Blue},
  urlcolor={Blue},
  pdfcreator={LaTeX via pandoc}}

\title{IDR4000 Mappeeksamen}
\author{Kandidatnr.}
\date{Invalid Date}

\begin{document}
\maketitle
\ifdefined\Shaded\renewenvironment{Shaded}{\begin{tcolorbox}[boxrule=0pt, enhanced, frame hidden, sharp corners, borderline west={3pt}{0pt}{shadecolor}, interior hidden, breakable]}{\end{tcolorbox}}\fi

\renewcommand*\contentsname{Innholdsfortegnelse}
{
\hypersetup{linkcolor=}
\setcounter{tocdepth}{2}
\tableofcontents
}
\bookmarksetup{startatroot}

\hypertarget{forord}{%
\chapter*{Forord}\label{forord}}
\addcontentsline{toc}{chapter}{Forord}

\markboth{Forord}{Forord}

Dette er en mappe som brukes til å lage en endelig pdf som skal leveres
på eksamen i IDR4000.

\bookmarksetup{startatroot}

\hypertarget{reliabilitet}{%
\chapter{Reliabilitet}\label{reliabilitet}}

\bookmarksetup{startatroot}

\hypertarget{vitenskapsfilosofi}{%
\chapter{Vitenskapsfilosofi}\label{vitenskapsfilosofi}}

\hypertarget{section}{%
\section{1.}\label{section}}

Hvilken forbindelse det er mellom en observasjon og teori som gjør det
mulig for oss å trekke slutninger i teorien basert på observasjonene,
regnes som et grunnleggende problem i vitenskapsfilosofien
(Godfrey-Smith 2003). Innledningsvis ønsker jeg å gi en forklaring av
begrepet induksjon. Begrepet innebærer at hva som har skjedd kan sies å
være forutsigende for hva som vil skje i framtiden. Samtidig innebærer
det muligheten til å generalisere et observert resultat fra tidligere
(Glass 2010).

Ifølge David Hume vil et hvert induktivt argument forutsette
uniformitetsprinsippet, ofte et skjult premiss ved induktive argumenter
at fremtiden vil alltid være lik som fortiden. Videre argumenterte han
at dette prinsippet ikke har noen rasjonell begrunnelse og dermed har
heller ingen av konklusjonene fra disse argumentene noen rasjonell
begrunnelse (Vassend 2023a). Han mente at vi ikke kunne bruke fortiden
til å predikere fremtiden. Fra et empiristisk syn som Hume representerte
så vil det å være rasjonell si å empirisk teste faktapåstander. Videre
skal jeg diskutere om konklusjonen til Hume kan være mulig å unngå.

For å belyse dette trekker jeg fram et eksempel fra Stove (1965) . Dette
er en flamme, en flamme har vært varm tidligere, derfor må også denne
flammen være varm (Stove 1965). Her har vi et premiss for å anta at
denne flammen også er varm, nettopp om at fremtiden vil være som
fortiden. Prinsippet er noe vi ikke nødvendigvis tenker over, men den må
være der for at vi skal kunne trekke antagelsen om flammen.
Uniformitetsprinsippet kan ikke bevises gjennom logikk, og faktisk så er
det jo heller slik at det er lett å forestille seg at fremtiden gjerne
er forskjellig fra fortiden (Vassend 2023b). Premisset ser ikke ut til å
være mulig å bevises til å være sant eller usant. I tillegg har vi en ny
antagelse om at uniformitetsprinsippet er sant for at argumentene våre
skal være bra. Med andre ord så ser vi ikke ut til å ha kommet utenom
induksjon fordi vi er nødt til å forutsette at fremtiden er lik som
fortiden, og bruker tidligere observasjoner til å si at det er slik.

Du kommer ikke utenom Hume sitt argument. Men likevel er det praktisk
for oss mennesker å godta denne usikkerheten. Det er heller ikke
irrasjonelt å godta uniformitetsprinsippet (Stove 1965). Men kanskje kan
det faktum at det heller ikke kan sies å være irrasjonelt faktisk gjør
at du rasjonelt kan begrunne bruken av induksjon? Og Popper var tross
alt heller ikke imot at vi brukte det i dagliglivet (Godfrey-Smith
2003). Likevel er det det at vi ikke kan blande følelser som er
rasjonelt med vitenskap. Og selv med nye tilnærminger så klarer vi ikke
komme helt utenom induksjon. Så vi klarer ikke helt å unngå Hume sin
konklusjon, selv om det i vitenskapen kan være praktisk å godta visse
premisser.

\hypertarget{section-1}{%
\subsection{2.}\label{section-1}}

Til kontrast fra Hume og induksjonsproblemet så var det senere en annen
vitenskapsfilosof, Karl Popper, som introduserte falsifikasjonisme.
Problemet var å skille hva som var vitenskap fra hva som ikke var
vitenskap, og Popper sin løsning på dette problemet ga han navnet
falsifikasjonisme (Godfrey-Smith 2003). Løsningen til Popper bygger på
at selv om man ikke kan bevise om noe kommer til å skje, så kan man med
sikkerhet si om noe faktisk skjedde eller ikke (Popper, Bartley, and
Popper 1982). Falsifikasjonisme sier videre at bekreftelse er en myte,
og at man bare kan bekrefte noe basert på å motbevise at noe ikke
stemmer (Godfrey-Smith 2003). Ifølge Popper var induktive resonnementer
en fiasko, men at logikk derimot var en god teori (Vassend 2023b).

Hypoteser skal testes og man skal prøve å falsifisere dem. Klarer man
ikke å falsifisere hypotesen holder den stand og man styrker teorien
sin. Helt til man får falsifisert den, da må man justere på teorien man
hadde. En teori er bedre hvis den er mer falsifiserbar, for å være det
må den være presis (Vassend 2023b). Det er nettopp det at man kan
falsifisere en hypotese med observasjoner som skiller et vitenskapelig
spørsmål fra spørsmål som ikke er vitenskapelige ifølge Popper.

Problemet som oppstår er at for å falsifisere noe så trenger du en
bekreftelse, noe som bare er mulig ved induksjon. For å bekrefte noe må
du basert på tidligere observasjoner, stole på det du har sett
tidligere, samt at det kan brukes til å si bekrefte. Vi har alstå et
stort problem; det er i mange tilfeller nødvendig med induksjon for å
falsifisere. Induksjonsproblemet er nettopp årsaken til at Popper ga opp
induksjon og forkastet det helt (Vassend 2023b). Dette problemet
beskriver også Popper selv (Godfrey-Smith 2003).

For å forklare tydeligere hva som menes kan vi bruke Newton sin lov om
tyngdekraft, F = ma. Her må man anta at tyngdekraften er konstant, og vi
er i gang med antagelser. Luftmotstanden er konstant, ny antagelse nok
en gang. Her er det involvert støttehypoteser som må være der for at
teorien skal være riktig. Duhem hadde et logisk poeng om at dersom det
predikerte resultatet skulle vise seg å være feil, så må det bety at
alle støttehypoteser og teori er feil fordi dette er det eneste logiske.
Enten er teorien feil, eller så er hjelpehypotesene feil. Problemet i
dette tilfellet blir at man må bekrefte at hjelpehypotesene er sanne og
at det dermed er teorien som er feil, eller så er det noe feil med
observasjonen vi har gjort. Vi har igjen induksjon (Vassend 2023b).

Hvis vi ser for oss en situasjon der du har to alternativer. En som er
blitt prøvd falsifisert mye og en som ikke har prøvd så mye, men er ny.
Ingen av teoriene har blitt falsifisert. Hvis man da tenker seg at man
velger det ene alternativet, kan man ikke ha det andre alternativet
(Vassend 2023b). Dersom jeg skulle tatt et slikt valg ville begrunnelsen
vært rasjonell. Jeg ville valgt det alternativet som etter fornuft
virker som det beste alternativet. Dette tenker jeg kunne vært basert på
sannsynlighet. Dermed har jeg igjen kommet tilbake til problemet, vi har
tatt en rasjonell beslutning. Samtidig er det etter Popper kommet andre
tilnærminger som prøver å løse induksjonsproblemet, som bayesianisme og
abduktivisme. Men likevel så lykkes man ikke helt med å komme utenom
det. Konklusjonen blir at man ikke klarer å komme helt utenom
induksjonsproblemet, og tilnærmingen til Popper kommer heller ikke helt
utenom problemet.

\hypertarget{referanser}{%
\section{Referanser}\label{referanser}}

\bookmarksetup{startatroot}

\hypertarget{ekstraksjon-og-analyse-av-protein}{%
\chapter{Ekstraksjon og analyse av
protein}\label{ekstraksjon-og-analyse-av-protein}}

\begin{center}\rule{0.5\linewidth}{0.5pt}\end{center}

\begin{center}\rule{0.5\linewidth}{0.5pt}\end{center}

\bookmarksetup{startatroot}

\hypertarget{studiedesign}{%
\chapter{Studiedesign}\label{studiedesign}}

\begin{center}\rule{0.5\linewidth}{0.5pt}\end{center}

\begin{center}\rule{0.5\linewidth}{0.5pt}\end{center}

\bookmarksetup{startatroot}

\hypertarget{analysere-eksperimenter-med-repeterte-forsuxf8k}{%
\chapter{Analysere eksperimenter med repeterte
forsøk}\label{analysere-eksperimenter-med-repeterte-forsuxf8k}}

\begin{center}\rule{0.5\linewidth}{0.5pt}\end{center}

\begin{center}\rule{0.5\linewidth}{0.5pt}\end{center}

\bookmarksetup{startatroot}

\hypertarget{referanseliste}{%
\chapter*{Referanseliste}\label{referanseliste}}
\addcontentsline{toc}{chapter}{Referanseliste}

\markboth{Referanseliste}{Referanseliste}

\hypertarget{refs}{}
\begin{CSLReferences}{1}{0}
\leavevmode\vadjust pre{\hypertarget{ref-glass2010}{}}%
Glass, David J. 2010. {``A Critique of the Hypothesis, and a Defense of
the Question, as a Framework for Experimentation.''} \emph{Clinical
Chemistry} 56 (7): 1080--85.
\url{https://doi.org/10.1373/clinchem.2010.144477}.

\leavevmode\vadjust pre{\hypertarget{ref-godfrey-smith2003}{}}%
Godfrey-Smith, Peter. 2003. \emph{Theory and Reality: An Introduction to
the Philosophy of Science}. Science and Its Conceptual Foundations.
Chicago: University of Chicago Press.

\leavevmode\vadjust pre{\hypertarget{ref-popper1982}{}}%
Popper, Karl R., William Warren Bartley, and Karl R. Popper. 1982.
\emph{Quantum Theory and the Schism in Physics}. The Postscript to the
Logic of Scientific Discovery / as Edited by w.w. Bartley, III. Totowa,
N.J: Rowan; Littlefield.

\leavevmode\vadjust pre{\hypertarget{ref-stove1965}{}}%
Stove, D. 1965. {``Hume, Probability, and Induction.''} \emph{The
Philosophical Review} 74 (2): 160.
\url{https://doi.org/10.2307/2183263}.

\leavevmode\vadjust pre{\hypertarget{ref-vassend2023a}{}}%
Vassend, Olav. 2023a. {``KvantMet: Vitenskapsfilosofi Dag 1,''} October.

\leavevmode\vadjust pre{\hypertarget{ref-vassend2023b}{}}%
---------. 2023b. {``KvantMet: Vitenskapsfilosofi Dag 2,''} October.

\end{CSLReferences}



\end{document}
